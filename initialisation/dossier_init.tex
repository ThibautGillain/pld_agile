%% Erläuterungen zu den Befehlen erfolgen unter
%% diesem Beispiel.

%% Article Template
\documentclass{scrartcl}

%% UTF8 Encoding
\usepackage[utf8]{inputenc}
\usepackage[T1]{fontenc}
\usepackage{lmodern}
\usepackage{subcaption}
\usepackage{setspace}

\setstretch{0.93}

%Tabellen mit fixen Breiten
\usepackage{tabularx}

%% Grafik
\usepackage{graphicx} 
\newcommand{\name}{Insta.edit}

%% Links im PDF
\usepackage{hyperref}



\title{Netzwerkfähiger Multi-User Texteditor\\
\textit{"\name"}}
\subtitle{Pflichtenheft}
\author{SEP WS 2017/18\\
Betreuer: Thomas Bock\\
Team 9\\ \\
Version 1.0}
\date{26.10.2017}

%TODO Header richtig formatieren

\begin{document}

\maketitle

\iffalse
\begin{figure}[h]
	\centering
  \includegraphics[width=0.3\textwidth]{../img/insta_logo}
	\label{fig:logo}
\end{figure}
\fi
\vfill

\begin{center}
  \begin{tabular}{ | l | r | }
    \hline
    Anselm Fehnker \\ \hline
    Augustin Bodet  \\ \hline
    Naoufel \\ \hline
    Thibualt Gilain \\ \hline
    Shuyao Shen \\ \hline
    Cedric Milinaire\\ \hline
    Martin Germain \\ \hline
  \end{tabular}
\end{center}

\thispagestyle{empty}
\pagebreak
\renewcommand{\contentsname}{Table des matières}
\tableofcontents
\newpage

\section{Fonctionnalités}
\subsection{Chargement fichier carte}
L'utilisateur a la possibilité de charger un fichier XML. Ce fichier représente une carte. Un message d'erreur s'affiche lorsque le fichier n'est pas de type XML. Lorsque le fichier correspond au critères la carte est afficher, et les données du fichier sont enregistrer par l'application. 
\subsection{Chargement fichier livraisons}
L'utilisateur a la possibilité de charger un fichier XML correspondant a une livraison. Un message d'erreur s'affiche lorsque le fichier n'est pas de type XML. Lorsque le fichier correspond au critères les données du fichier sont enregistrer et afficher sur la carte. Si aucune carte est chargé un message d'erreur est envoyé. 
\subsection{Calcul}
Lorsqu'une carte et une livraison est chargé l'utilisateur a la possibilité de calcule une tournée. La tournée est calculé avec les donné recu par les deux fichiers XML. Une fois calculé la tournée est affiché. 
\subsection{Afficher liste}
Le système calcule le plus court chemin pour le livreur en fonction de son adresse de départ, les points d’enlèvement et de livraison, pour qu'il pourra en suite affiche la liste des adresses où le livreur doit enlever ou livrer un colis, avec les heures d'arrivée et de départ prévues.
\subsection{Modifier tournée}
L'utilisateur peut faire des modifications (supprimer les livraisons, ajouter de nouvelles livraisons, ou changer l'ordre de passage). Le système recalcule le passage et le met à jour. L'utilisateur a le droit d'annuler ses modifications à tout moment.


\end{document}