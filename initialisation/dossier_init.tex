%% Erläuterungen zu den Befehlen erfolgen unter
%% diesem Beispiel.

%% Article Template
\documentclass{scrartcl}

%% UTF8 Encoding
\usepackage[utf8]{inputenc}
\usepackage[T1]{fontenc}
\usepackage{lmodern}
\usepackage{subcaption}
\usepackage{setspace}

\setstretch{0.93}

%Tabellen mit fixen Breiten
\usepackage{tabularx}

%% Grafik
\usepackage{graphicx} 
\newcommand{\name}{Insta.edit}

%% Links im PDF
\usepackage{hyperref}



\title{Netzwerkfähiger Multi-User Texteditor\\
\textit{"\name"}}
\subtitle{Pflichtenheft}
\author{SEP WS 2017/18\\
Betreuer: Thomas Bock\\
Team 9\\ \\
Version 1.0}
\date{26.10.2017}

%TODO Header richtig formatieren

\begin{document}

\maketitle

\iffalse
\begin{figure}[h]
	\centering
  \includegraphics[width=0.3\textwidth]{../img/insta_logo}
	\label{fig:logo}
\end{figure}
\fi
\vfill

\begin{center}
  \begin{tabular}{ | l | r | }
    \hline
    Anselm Fehnker \\ \hline
    Augustin Bodet  \\ \hline
    Naoufel \\ \hline
    Thibualt Gilain \\ \hline
    Shuyao Shen \\ \hline
    Cedric Milinaire\\ \hline
    Martin Germain \\ \hline
  \end{tabular}
\end{center}

\thispagestyle{empty}
\pagebreak
\renewcommand{\contentsname}{Table des matières}
\tableofcontents
\newpage

\section{Fonctionnalités}
\subsection{Chargement fichiers}
\subsection{Calcul}
\subsection{Afficher liste}
\subsection{Modifier tournée}


\end{document}